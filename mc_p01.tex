\title{Excercise 1}
\author{\Large
	\textsc{Markus Korpinen} \\
	\mbox{}\\
	Basics if Monte Carlo -simulations\\
	Department of Physics, University of Helsinki
}
\date{\today}

\documentclass[12pt]{article}

\usepackage{graphicx}
\DeclareGraphicsExtensions{.pdf,.png,.jpg}

\usepackage[utf8]{inputenc}
\usepackage[T1]{fontenc}
\usepackage[paper=a4paper,dvips,top=2.5cm,left=2.5cm,right=2.5cm,
    foot=1cm,bottom=4cm]{geometry}
\usepackage[fleqn]{amsmath}

% Various AMS fonts.
\usepackage{amsfonts}

% Many special mathematical characters, including the
% famous blackboard bold letters used.
\usepackage{amssymb}

% Theorems using the AMS style.
\usepackage{amsthm}
% The following is probably the optimal method for numbering
% lemmas, examples, definitions their like. We number them
% all together. It is annoying and difficult
% for the reader to search for these theorem-like entities in
% if they are.

\renewcommand{\textfraction}{0.01}
% Numbering of theorems is by section, e.g., Theorem 1.3, etc.
% This makes it easier for the reader to search for them.

\newtheorem{theorem}{Theorem}[section]
\newtheorem{definition}[theorem]{Definition}
\newtheorem{lemma}[theorem]{Lemma}
\newtheorem{corollary}[theorem]{Corollary}
\newtheorem{fact}[theorem]{Fact}
\newtheorem{example}[theorem]{Example}

% If the paper has a large number of equations, figures, tables, etc.,
% then they should be numbered within sections. Comment out
% if this is not what you want.
\numberwithin{equation}{section}
%\numberwithin{figure}{section}
\numberwithin{table}{section}


%\usepackage{showkeys}

\begin{document}


\maketitle

\newpage{}

%\tableofcontents{}

\section*{1. problem}

\begin{equation*}
P_{hit}=P_1*P_2,
\end{equation*}
where $P_1$ is the probability that center of needle is less than $l/2$ away from line and $P_2$ is the the probability that the needle is in a right angle to cross the line. Probabilities $P_1$ and $P_2$ are depending on each other so they have to be multiplied. Because $P_1$ can fall to eather side of line 
\begin{equation*}
P_1=\frac{l/2+l/2}{d}=\frac{l}{d}.
\end{equation*}
Probability that needle crosses the line when center falls x away is a function of x. The probability $P_2$ when needle drops somewhere can be obtained by integrating and setting $l=2$ (unit circle)
\begin{equation*}
P_2=4\int_{0}^{1}\frac{cos^{-1}(x)}{2\pi}dx=\frac{2}{\pi}.
\end{equation*}
Therefore
\begin{equation*}
P_{hit}=P_1*P_2=\frac{2l}{\pi d}.
\end{equation*}
\subsection*{a)}Figure shows that when you throw only 10 times the predicted value for $\pi$ is bad.
\includegraphics[width=15cm]{mc_p01_a.jpg}
\subsection*{a)}You can see from figure that after approximately 5000 throws result starts to converge.
\includegraphics[width=15cm]{mc_p01_b.jpg}
\end{document}